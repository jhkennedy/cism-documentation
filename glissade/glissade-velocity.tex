
\section{Velocity solver}
\label{sc:glissade-velocity}

Glissade computes the ice velocity by solving an appropriate
approximation of the Stokes equations, given the 2D surface elevation and thickness fields and
the 3D temperature field.
Section (cite section) describes assembly and solution of the matrix problem for the 3D first-order
Blatter-Pattyn approximation.  Section (cite section) then discusses assembly and solution
methods for simpler approximations, including the SIA and SSA.

\subsection{Blatter-Pattyn approximation}

The basic equations of the Blatter-Pattyn approximation in Cartesian coordinates, repeated from Section (cite section), are 

\begin{equation}
  \label{gliss.eq.stress_balance}
  \begin{split}
    x: \quad \frac{\partial }{\partial x}\left( 4 \eta \frac{\partial u}{\partial x} +  2 \eta \frac{\partial v}{\partial y} \right) + \frac{\partial }{\partial y}\left[ \eta \left( \frac{\partial u}{\partial y} + \frac{\partial v}{\partial x} \right) \right]+\frac{\partial }{\partial z}\left( \eta \frac{\partial u}{\partial z} \right) = \rho g\frac{\partial s}{\partial x}, \\
    y: \quad \frac{\partial }{\partial y}\left( 4 \eta \frac{\partial v}{\partial y} +  2 \eta \frac{\partial u}{\partial x} \right) + \frac{\partial }{\partial x}\left[ \eta \left( \frac{\partial u}{\partial y} + \frac{\partial v}{\partial x} \right) \right]+\frac{\partial }{\partial z}\left( \eta \frac{\partial v}{\partial z} \right) = \rho g\frac{\partial s}{\partial y},  \\
  \end{split}
\end{equation}

\noindent
where $u$ and $v$ are the components of horizontal velocity, $\eta$ is the effective viscosity, $s$ is the ice surface elevation,
$\rho$ is the density of ice (assumed constant), and $g$ is gravitational acceleration.  

As in Glide, the equations are discretized on two structured, rectangular horizontal meshes: an unstaggered mesh
of dimension $(nx,ny)$ and a staggered mesh of dimension $(nx-1,ny-1)$.  The rectangles
are called \textit{cells}, and the corners of each cell (where four rectangles meet) are called \textit{vertices}.
The vertical levels of the mesh are based on a terrain-following sigma coordinate system (insert sigma equation).
There are $nz$ \textit{levels} in the vertical, with $nz-1$ \textit{layers} between these levels.
An \textit{element} is the region associated with a particular cell and layer; thus there are
$(nx)(ny)(nz-1)$ elements on the mesh.  A \textit{node} is a point where eight elements intersect (or where four elements
intersect at the upper and lower surface). There are $(nx-1)(ny-1)(nz)$ nodes on the mesh.

Scalar 2D fields such as ice thickness $H$ and surface elevation $s$ are defined for each cell.
Scalar 3D fields such as ice temperature $T$ lie at the center of each element (i.e., in each layer
associated with each cell). Gradients of 2D scalar fields (e.g., the surface slope $\nabla s$) are defined at vertices.
The velocity components live at nodes.

The effective viscosity is defined by

\begin{equation}
  \eta \equiv \frac{1}{2} A^{\frac{-1}{n}} \dot{\varepsilon }_{e}^{\frac{1-n}{n}},
\end{equation}

\noindent
where $A$ is the temperature-dependent flow factor in Glen's law, and $\dot{\varepsilon }_{e}$ is the effective strain rate,
given in the Blatter-Pattyn approximation by 

\begin{equation}
{{\dot{\varepsilon }}^{2}}_{e}={{\dot{\varepsilon }}^{2}}_{xx}+{{\dot{\varepsilon }}^{2}}_{yy}+{{\dot{\varepsilon }}_{xx}}{{\dot{\varepsilon }}_{yy}}+{{\dot{\varepsilon }}^{2}}_{xy}+{{\dot{\varepsilon }}^{2}}_{xz}+{{\dot{\varepsilon }}^{2}}_{yz}.
\end{equation}

\noindent
Both $A$ and $\dot{\varepsilon }_{e}$ are 3D fields co-located with the temperature $T$.

Given $T$, $s$, $H$, and an initial guess for $u$ and $v$, the problem is to solve Eq. \eqref{gliss.eq.stress_balance}
for $u$ and $v$.  This problem can be written in the form

\begin{equation}
  \mathbf{A} \mathbf{x} = \mathbf{b},
\end{equation}

\noindent
or more fully,

\begin{equation}
  \label{gliss.eq.matrix}
  \begin{matrix}
    \left[ \begin{matrix}
        \mathbf{A}_{\mathbf{uu}} & \mathbf{A}_{\mathbf{uv}}  \\
        \mathbf{A}_{\mathbf{vu}} & \mathbf{A}_{\mathbf{vv}}  \\
      \end{matrix} \right]\left[ \begin{matrix}
        \mathbf{u}  \\
        \mathbf{v}  \\
      \end{matrix} \right]=\left[ \begin{matrix}
        \mathbf{b}_{\mathbf{u}}  \\
        \mathbf{b}_{\mathbf{v}}  \\
      \end{matrix} \right], \\ 
    \\ 
    \mathbf{A}_{\mathbf{uu}}\mathbf{u} + \mathbf{A}_{\mathbf{uv}}\mathbf{v} =\mathbf{b}_{\mathbf{u}},
    \quad \quad \mathbf{A}_{\mathbf{vu}}\mathbf{u} + \mathbf{A}_{\mathbf{vv}}\mathbf{v} =\mathbf{b}_{\mathbf{v}}. \\ 
  \end{matrix}
\end{equation}

\noindent
Eq. \eqref{gliss.eq.matrix} explicitly shows the four parts of the total matrix $\mathbf{A}$,
with the $u$ and $v$ components of the solution and right-hand-side vectors separated into two subvectors.
In Glissade, $\mathbf{A}$ is always symmetric and positive-definite.

Since $\mathbf{A}$ depends (through $\eta$) on $u$ and $v$, the problem is nonlinear and must be solved iteratively
as described in Section (insert section ref).  For a given nonlinear iteration, we compute $\eta$ based on the
current guess for the velocity field, then solve a linear problem of the form \eqref{gliss.eq.matrix}.
Then we update $\eta$ and repeat until the solution converges to within a given tolerance.  This procedure
is known as Picard iteration.

The following sections describe how the matrix equations are assembled and solved. 

\subsubsection{Assembly}


