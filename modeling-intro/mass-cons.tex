\subsection{Conservation of Mass}

In this case, $\phi$ from our general conservation equation represents the mass $M$, or more conveniently
$M = \int_V \rho dV$, the integral of the density over the volume. Assuming that there are no sources or 
sinks of mass in the volume ($H$=0), the mass conservation equation is written

\begin{equation}
\int_{V}\frac{\partial \rho} {\partial t} ~dV ~+~ \int_{V} \nabla \cdot \rho \mathbf{u} dV~=~0
\label{eq:mascon1}
\end{equation}

Ice is incompressible (the density does not change in time), which provides the equation for 
local mass continuity,

\begin{equation}
\nabla \cdot \mathbf{u} ~=~0.
\label{eq:mascon2}
\end{equation}

Equation \eqref{eq:mascon2} says that the velocity field is ``divergence free". Applying the $\nabla$ 
operator in cartesian coordinates gives

\begin{equation}
\frac{\partial u_{x}}{\partial x}~+~\frac{\partial u_{y}}{\partial y} ~+\frac{\partial u_{z}}{\partial z}~=~0.  
\label{eq:mascon3}
\end{equation}

To make use of this statement, we need to integrate from the base $b$ to the upper surface $s$ of the ice mass,

\begin{equation}
\int_{b}^{s} \left( \frac{\partial u_{x}}{\partial x}~+~\frac{\partial u_{y}}{\partial y} ~+\frac{\partial u_{z}}{\partial z}\right) dz~=~0.  
\label{eq:mascon4}
\end{equation}

The integral of $\frac{\partial u_z}{\partial z}$ is simply the difference between
the vertical component of the velocity at the upper and lower surfaces, so

\begin{equation}
u_{z} \left(s\right)-u_{z} \left(b\right)~=~-\int_{b}^{s} \frac{\partial u_{x}}{\partial x} dz ~-~\int_{b}^{s} \frac{\partial u_{y}}{\partial y} dz  
\label{eq:mascon5}
\end{equation}

Changing the order of integration and differentiation using the Leibnitz rule we obtain

\begin{equation}
\begin{matrix}
u_{z} \left(s\right)-u_{z} \left(b\right) & = & -~\frac{\partial}{\partial x} \int_{b}^{s} u_{x} dz ~ +~u_{x}(s)\frac{\partial s}{\partial x} ~-~ u_{x}(b)\frac{\partial b}{\partial x}  \\ 
& & -~\frac{\partial}{\partial y}\int_{b}^{s} u_{y} dz   ~ +~u_{y}(s)\frac{\partial s}{\partial y} ~-~ u_{y}(b)\frac{\partial b}{\partial x}.
\end{matrix}
\label{eq:mascon6}
\end{equation}

The vertical velocity at the upper surface $u_{z}(s)$ is the result of motion parallel to the surface slope, 
the rate of new ice accumulation $\dot{a}$, and any time-change in the surface height,

\begin{equation}
u_{z} \left(s\right)~=~\frac{\partial s}{\partial t}~+~u_{x}(s)\frac{\partial s}{\partial x}~+~u_{y}(s)\frac{\partial s}{\partial y}~-~\dot{a}, 
\label{eq:mascon7}
\end{equation}

recognizing that a negative accumulation rate indicates ablation. Similarly, the vertical velocity at the lower surface is

\begin{equation}
u_{z} \left(b\right)~=~\frac{\partial b}{\partial t}~+~u_{x}(b)\frac{\partial b}{\partial x}~+~u_{y}(b)\frac{\partial b}{\partial y}~-~\dot{b} 
\label{eq:mascon8}
\end{equation}

in which $\dot{b}$ represents the basal accumulation rate.

Substituting equations \eqref{eq:mascon7} and \eqref{eq:mascon8} into \eqref{eq:mascon6} we find that many terms cancel,

\begin{equation}
\frac{\partial s}{\partial t}~-~\dot{a}~-~\frac{\partial b}{\partial t}~+~\dot{b}~=~-~\frac{\partial}{\partial x} \int_{b}^{s} u_{x} dz~-~\frac{\partial}{\partial y} \int_{b}^{s} u_{y} dz.
\label{eq:mascon9}
\end{equation}

Finally, making the substitution that the ice thickness $h=s-b$ we obtain

\begin{equation}
\frac{\partial h}{\partial t}~=~-~\frac{\partial}{\partial x} \int_{b}^{s} u_{x} dz~-~\frac{\partial}{\partial y} \int_{b}^{s} u_{y} dz ~+~\dot{a}~-~\dot{b}.
\label{eq:mascon10}
\end{equation}

If we integrate in the vertical, we get 

\begin{equation}
\frac{\partial h}{\partial t}~=~-~\nabla \cdot \left( U_{i} h \right) +~\dot{a}~-~\dot{b}
\label{eq:mascon11}
\end{equation}

in which $U_{i}$ represents the vertically averaged velocity, i.e. $U_i = \frac{1}{h}\int_{b}^{s}u_{i}dz$ . Note that this equation is 
prognostic; we can use the current velocity and geometry of the ice to compute a rate of change in the geometry.
