\section{Conservation of Mass}


In this case $\phi$ is the mass $M$, or more conveniently
$M = \int_V \rho dV$, the integral of the density over the volume.
Assuming that there are no sources or sinks of mass in the volume ($H$ =
0), the conservation equation is written

\begin{equation}
\int_{V}\frac{\partial \rho} {\partial t} ~dV ~+~ \int_{V} \nabla \cdot \rho \mathbf{u} dV~=~0
\end{equation}

Ice is incompressibile, meaning that the density does not change in
time, and the equation for local mass continuity is

\begin{equation}
\nabla \cdot \mathbf{u} ~=~0.
\end{equation}

Applying the $\nabla$ operator in a cartesian coordinates produces

\begin{equation}
\frac{\partial u_{x}}{\partial x}~+~\frac{\partial u_{y}}{\partial y} ~+\frac{\partial u_{z}}{\partial z}~=~0.  
\end{equation}

To make use of this statement, we need to integrate

\begin{equation}
\int_{b}^{s} \left( \frac{\partial u_{x}}{\partial x}~+~\frac{\partial u_{y}}{\partial y} ~+\frac{\partial u_{z}}{\partial z}\right) dz~=~0  
\end{equation}

from the base $b$ to the upper surface $s$ of the ice mass. The integral
of $\frac{\partial u_z}{\partial z}$ is simply the difference between
the vertical component of the velocity at the upper and lower surfaces,
so

\begin{equation}
u_{z} \left(s\right)-u_{z} \left(b\right)~=~-\int_{b}^{s} \frac{\partial u_{x}}{\partial x} dz ~-~\int_{b}^{s} \frac{\partial u_{y}}{\partial y} dz  
\end{equation}

Changing the order of integration using Leibnitz rule

\begin{equation}
\begin{matrix}
u_{z} \left(s\right)-u_{z} \left(b\right) & = & -~\frac{\partial}{\partial x} \int_{b}^{s} u_{x} dz ~ +~u_{x}(s)\frac{\partial s}{\partial x} ~-~ u_{x}(b)\frac{\partial b}{\partial x}  \\ 
& & -~\frac{\partial}{\partial y}\int_{b}^{s} u_{y} dz   ~ +~u_{y}(s)\frac{\partial s}{\partial y} ~-~ u_{y}(b)\frac{\partial b}{\partial x}
\end{matrix}
\end{equation}

The vertical velocity at the upper surface is the result of motion down
the surface slope, the rate of new accumulation $\dot{a}$ and any
time-change in surface height

\begin{equation}
u_{z} \left(s\right)~=~\frac{\partial s}{\partial t}~+~u_{x}(s)\frac{\partial s}{\partial x}~+~u_{y}(s)\frac{\partial s}{\partial y}~-~\dot{a} 
\end{equation}

recognizing that a negative accumulation rate indicates ablation.
Similarly, the vertical velocity at the lower surface is

\begin{equation}
u_{z} \left(b\right)~=~\frac{\partial b}{\partial t}~+~u_{x}(b)\frac{\partial b}{\partial x}~+~u_{y}(b)\frac{\partial b}{\partial y}~-~\dot{b} 
\end{equation}

in which $\dot{b}$ represents the basal accumulation rate.

Substituting equations we find that many terms cancel

\begin{equation}
\frac{\partial s}{\partial t}~-~\dot{a}~-~\frac{\partial b}{\partial t}~+~\dot{b}~=~-~\frac{\partial}{\partial x} \int_{b}^{s} u_{x} dz~-~\frac{\partial}{\partial y} \int_{b}^{s} u_{y} dz
\end{equation}

Finally, making the simplification $h=s-b$ we have

\begin{equation}
\frac{\partial h}{\partial t}~=~-~\frac{\partial}{\partial x} \int_{b}^{s} u_{x} dz~-~\frac{\partial}{\partial y} \int_{b}^{s} u_{y} dz ~+~\dot{a}~-~\dot{b}
\end{equation}

The vertically-integrated form

\begin{equation}
\frac{\partial h}{\partial t}~=~-~\nabla \cdot \left( U_{i} h \right) +~\dot{a}~-~\dot{b}
\end{equation}

in which $U_{i}$ represents the vertically averaged velocity, i.e.
$U_i = \frac{1}{h}\int_{b}^{s}u_{i}dz$ . This equation is prognostic. We
use the current geometry of the ice to compute a future time-change in
that geometry.
